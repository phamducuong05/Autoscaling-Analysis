\chapter{Kết luận}

\section{Tổng kết kết quả}

Dự án Autoscaling Analysis đã hoàn thành thành công, triển khai ba mô hình dự báo: \textbf{ARIMA} (thống kê), \textbf{LSTM} (Deep Learning), và \textbf{XGBoost} (Machine Learning) trên ba khung thời gian (1m, 5m, 15m) với bộ dữ liệu NASA HTTP Logs (1/7 - 31/8/1995).

Kết quả cho thấy sự vượt trội của các mô hình hiện đại. \textbf{XGBoost} đạt hiệu quả tốt nhất: RMSE = 41.94, MAE = 31.92, MAPE = 26.48\%, cải thiện 65.6\% RMSE so với ARIMA. \textbf{LSTM} xếp thứ hai: RMSE = 44.59, MAE = 33.86, MAPE = 26.91\%, cải thiện 63.4\% RMSE so với ARIMA. \textbf{ARIMA} xếp cuối với RMSE = 121.86, MAE = 99.20, MAPE = 86.26\%, chỉ cải thiện 0.5\% so với baseline do thiếu thành phần mùa vụ.

Về autoscaling, chiến lược \textbf{Predictive scaling} đạt hiệu quả tốt nhất với tổng chi phí \$375.12, tiết kiệm 47.9\% so với static allocation (10 server), chỉ 0.9\% thời gian bị overload.

\section{Hạn chế và hướng phát triển}

\textbf{Hạn chế:} Chỉ test trên dữ liệu lịch sử 1995, chưa triển khai production, ARIMA thiếu thành phần mùa vụ, chưa phát hiện anomaly, chỉ tập trung vào 3 khung thời gian định sẵn.

\textbf{Hướng phát triển:} Triển khai SARIMA để nắm bắt mùa vụ; ensemble methods kết hợp ARIMA, LSTM, XGBoost; tích hợp streaming data (Kafka/Flink) cho real-time; tích hợp cloud APIs (AWS/GCP/Azure) để autoscaling tự động; phát hiện anomaly (Isolation Forest, One-Class SVM); retraining định kỳ để thích ứng concept drift; monitoring và alerting cho các chỉ số hiệu quả.
