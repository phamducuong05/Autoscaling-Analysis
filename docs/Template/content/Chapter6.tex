\chapter{Đánh giá mô hình}

\section{Bảng so sánh tổng hợp}

Sau khi triển khai và huấn luyện ba mô hình dự báo khác nhau (ARIMA, LSTM, và XGBoost) cho bài toán dự báo lưu lượng truy cập web, chúng tôi thực hiện đánh giá toàn diện để so sánh hiệu quả của từng mô hình. Việc so sánh này được thực hiện trên cùng một tập kiểm tra (test set) từ 23/8/1995 đến 31/8/1995 với cửa sổ thời gian 5 phút, đảm bảo tính công bằng và nhất quán trong đánh giá.

Bảng \ref{tab:comparison_all} tóm tắt kết quả đánh giá của ba mô hình trên bốn chỉ số quan trọng: RMSE (Root Mean Squared Error), MAE (Mean Absolute Error), MAPE (Mean Absolute Percentage Error), và tỷ lệ overfitting (tỷ lệ giữa RMSE out-of-sample và RMSE in-sample).

\begin{table}[H]
    \centering
    \caption{So sánh tổng hợp hiệu quả dự báo của ba mô hình cho cửa sổ 5 phút}
    \label{tab:comparison_all}
    \begin{tabular}{lccccc}
        \toprule
        Mô hình & RMSE & MAE & MAPE (\%) & Tỷ lệ overfitting & Xếp hạng \\
        \midrule
        ARIMA(2,0,1) & 121.86 & 99.20 & 86.26 & 2.48x & 3 \\
        LSTM & 44.59 & 33.86 & 26.91 & $\sim$1.2x & 2 \\
        XGBoost & 41.94 & 31.92 & 26.48 & 1.11x & 1 \\
        \bottomrule
    \end{tabular}
\end{table}

Dựa trên kết quả trong Bảng \ref{tab:comparison_all}, chúng tôi có thể sắp xếp các mô hình theo hiệu quả dự báo trên từng chỉ số như sau. Về chỉ số \textbf{RMSE}: XGBoost đạt RMSE thấp nhất (41.94), xếp hạng 1; LSTM đứng thứ hai với RMSE = 44.59; ARIMA xếp hạng cuối với RMSE = 121.86. Về chỉ số \textbf{MAE}: XGBoost cũng dẫn đầu với MAE = 31.92; LSTM xếp thứ hai với MAE = 33.86; ARIMA xếp hạng cuối với MAE = 99.20. Về chỉ số \textbf{MAPE}: XGBoost tiếp tục dẫn đầu với MAPE = 26.48\%; LSTM xếp thứ hai với MAPE = 26.91\%; ARIMA xếp hạng cuối với MAPE = 86.26\%. Về chỉ số \textbf{Tỷ lệ overfitting}: XGBoost có tỷ lệ overfitting thấp nhất (1.11x), cho thấy khả năng tổng quát hóa tốt nhất; LSTM có tỷ lệ overfitting khoảng 1.2x; ARIMA có tỷ lệ overfitting cao nhất (2.48x), cho thấy hiện tượng overfitting nghiêm trọng.

Bảng \ref{tab:improvement} tóm tắt mức cải thiện của XGBoost và LSTM so với ARIMA trên từng chỉ số.

\begin{table}[H]
    \centering
    \caption{Mức cải thiện của XGBoost và LSTM so với ARIMA}
    \label{tab:improvement}
    \begin{tabular}{lccc}
        \toprule
        Chỉ số & XGBoost vs ARIMA & LSTM vs ARIMA & XGBoost vs LSTM \\
        \midrule
        RMSE & +65.6\% & +63.4\% & +5.9\% \\
        MAE & +67.8\% & +65.9\% & +5.7\% \\
        MAPE & +69.3\% & +68.8\% & +1.6\% \\
        Overfitting & -55.2\% & -51.6\% & -7.5\% \\
        \bottomrule
    \end{tabular}
\end{table}

Kết quả cho thấy cả XGBoost và LSTM đều cải thiện đáng kể về độ chính xác dự báo so với ARIMA, với mức cải thiện khoảng 63-69\% trên các chỉ số RMSE, MAE và MAPE. Trong đó, XGBoost có hiệu quả tương đương với LSTM, với lợi thế nhỏ về RMSE (5.9\% tốt hơn), MAE (5.7\% tốt hơn) và MAPE (1.6\% tốt hơn). Đặc biệt, XGBoost có tỷ lệ overfitting thấp nhất (1.11x), cho thấy khả năng tổng quát hóa tốt nhất trong ba mô hình.

\section{Lựa chọn mô hình tối ưu}

Để lựa chọn mô hình tối ưu cho bài toán dự báo lưu lượng truy cập web, chúng tôi xác định các tiêu chí quan trọng sau: (1) \textbf{Độ chính xác dự báo}: Mô hình phải có độ chính xác cao trên các chỉ số RMSE, MAE và MAPE; (2) \textbf{Khả năng tổng quát hóa}: Mô hình phải có tỷ lệ overfitting thấp, đảm bảo hiệu quả tốt trên dữ liệu mới; (3) \textbf{Tốc độ huấn luyện}: Mô hình phải có thời gian huấn luyện hợp lý để có thể retraining thường xuyên; (4) \textbf{Khả năng giải thích}: Mô hình phải có khả năng giải thích để hiểu rõ các yếu tố ảnh hưởng đến dự báo; (5) \textbf{Khả năng triển khai}: Mô hình phải dễ triển khai trong môi trường production với yêu cầu tài nguyên hợp lý; và (6) \textbf{Độ tin cậy}: Mô hình phải hoạt động ổn định và đáng tin cậy trong thời gian dài.

Dựa trên các tiêu chí này, chúng tôi đánh giá ba mô hình trên thang điểm từ 1 đến 5 (1 = kém nhất, 5 = tốt nhất) như trong Bảng \ref{tab:criteria_scoring}.

\begin{table}[H]
    \centering
    \caption{Đánh giá ba mô hình theo các tiêu chí lựa chọn}
    \label{tab:criteria_scoring}
    \begin{tabular}{lccccc}
        \toprule
        Tiêu chí & ARIMA & LSTM & XGBoost & Trọng số & Điểm trọng số \\
        \midrule
        Độ chính xác dự báo & 1 & 4 & 5 & 0.30 & 2.7 \\
        Khả năng tổng quát hóa & 1 & 4 & 5 & 0.25 & 2.5 \\
        Tốc độ huấn luyện & 5 & 2 & 4 & 0.15 & 1.8 \\
        Khả năng giải thích & 5 & 1 & 4 & 0.15 & 1.5 \\
        Khả năng triển khai & 5 & 2 & 4 & 0.10 & 1.0 \\
        Độ tin cậy & 3 & 4 & 5 & 0.05 & 0.4 \\
        \midrule
        Tổng điểm & 3.10 & 3.10 & 4.80 & - & - \\
        \bottomrule
    \end{tabular}
\end{table}

Kết quả đánh giá cho thấy XGBoost đạt tổng điểm cao nhất (4.80), vượt trội hơn đáng kể so với ARIMA (3.10) và LSTM (3.10). ARIMA và LSTM có tổng điểm bằng nhau, nhưng ARIMA mạnh về khả năng giải thích và triển khai trong khi LSTM mạnh về độ chính xác và khả năng tổng quát hóa.