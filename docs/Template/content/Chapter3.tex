\chapter{Mô hình ARIMA}

\section{Tổng quan mô hình ARIMA}

Mô hình ARIMA (AutoRegressive Integrated Moving Average) kết hợp ba thành phần: \textbf{AR} sử dụng giá trị quá khứ, \textbf{I} áp dụng sai phân để biến đổi thành chuỗi dừng, và \textbf{MA} sử dụng sai số dự báo quá khứ.

\section{Phương pháp huấn luyện}

\subsection{Đánh giá tính dừng và lựa chọn tham số}

Chuỗi thời gian có tính dừng (p-value ADF = 0.000000), không cần sai phân (d = 0). Chúng tôi sử dụng \texttt{auto\_arima} để tìm kiếm tham số tối ưu dựa trên tiêu chuẩn AIC. Kết quả cho ba cửa sổ thời gian: \texttt{ARIMA(5,0,1)} cho 1m (AIC = 605677.82), \texttt{ARIMA(2,0,1)} cho 5m (AIC = 159089.09), và \texttt{ARIMA(3,0,2)} cho 15m (AIC = 62299.72).

\begin{table}[H]
    \centering
    \caption{Tham số ARIMA tối ưu cho ba cửa sổ thời gian}
    \label{tab:arima_params}
    \begin{tabular}{lcccc}
        \toprule
        Cửa sổ & ARIMA(p,d,q) & AIC & BIC & Số lượng huấn luyện \\
        \midrule
        1m & (5,0,1) & 605677.82 & 605751.48 & 74,880 \\
        5m & (2,0,1) & 159089.09 & 159127.16 & 14,976 \\
        15m & (3,0,2) & 62299.72 & 62341.01 & 4,992 \\
        \bottomrule
    \end{tabular}
\end{table}

Hình \ref{fig:acf_pacf} cho thấy đồ thị ACF và PACF cho cửa sổ 5m. Đồ thị PACF có đỉnh rõ rệt tại lag 1 và lag 2, gợi ý mô hình ARIMA(2,0,1).

\begin{figure}[H]
    \centering
    \includegraphics[width=0.9\textwidth]{images/chapter3_acf_pacf.png}
    \caption{Đồ thị ACF và PACF cho phân tích tham số mô hình}
    \label{fig:acf_pacf}
\end{figure}

\section{Kết quả cho cửa sổ 5 phút}

\subsection{Cấu trúc mô hình}

Mô hình \texttt{ARIMA(2,0,1)} được huấn luyện trên 14,976 quan sát. Các hệ số: \texttt{$\phi_1 = 1.228, \phi_2 = -0.234, \theta_1 = -0.771$}. Tất cả tham số đều có ý nghĩa thống kê (p-value < 0.001).

\subsection{Hiệu quả dự báo}

Trên tập kiểm tra (2,592 quan sát): RMSE = 121.86, MAE = 99.20, MAPE = 86.26\%. So với baseline, ARIMA chỉ cải thiện 0.5\% về RMSE, cho thấy giá trị gia tăng hạn chế.

\begin{table}[H]
    \centering
    \caption{Hiệu quả dự báo ARIMA cho cửa sổ 5 phút}
    \label{tab:performance_5m}
    \begin{tabular}{lcccc}
        \toprule
        Cửa sổ & MSE & RMSE & MAE & MAPE (\%) \\
        \midrule
        5m & 14849.92 & 121.86 & 99.20 & 86.26 \\
        \bottomrule
    \end{tabular}
\end{table}

\section{Phân tích và đánh giá}

\subsection{So sánh với baseline}

ARIMA chỉ cải thiện 0.5\% so với mean baseline (RMSE: 121.86 vs 122.50). Hình \ref{fig:baseline_comparison} cho thấy dự báo ARIMA gần như trùng với dự báo mean.

\begin{table}[H]
    \centering
    \caption{So sánh mô hình ARIMA với các baseline cho cửa sổ 5 phút}
    \label{tab:baseline_comparison}
    \begin{tabular}{lcccc}
        \toprule
        Cửa sổ & Naive RMSE & Mean RMSE & ARIMA RMSE & Cải thiện (\%) \\
        \midrule
        5m & 138.74 & 122.50 & 121.86 & +0.5 \\
        \bottomrule
    \end{tabular}
\end{table}

\begin{figure}[H]
    \centering
    \includegraphics[width=0.9\textwidth]{images/chapter3_baseline_comparison_5m.png}
    \caption{So sánh mô hình ARIMA với các baseline cho cửa sổ 5 phút}
    \label{fig:baseline_comparison}
\end{figure}

\subsection{Phân tích overfitting}

ARIMA có hiện tượng overfitting nghiêm trọng: RMSE out-of-sample (121.86) gấp 2.48 lần RMSE in-sample (49.02). Điều này cho thấy mô hình phù hợp tốt với dữ liệu huấn luyện nhưng tổng quát hóa kém trên dữ liệu mới.
